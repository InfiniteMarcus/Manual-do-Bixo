\section{VIDA ACADÊMICA}
\subsection{Sobrevivendo em Sorocaba}
Nesta seção colocaremos algumas informações importantes para que você sobreviva da melhor maneira na cidade de Sorocaba; não se deixe levar pelo título dessa seção! Viver em Sorocaba não é uma coisa do outro mundo, mas algumas informações prévias facilitarão muito sua vida.

\subsubsection{Onde Morar}
Se você, como muitos outros estudantes da universidade, não é da cidade de Sorocaba e ainda não conseguiu um lugar para morar, você pode pesquisar algo no grupo de repúblicas do Facebook <\texttt{fb.com/groups/208040842611873/}>. Lá muitos veteranos divulgam vagas em suas repúblicas.

Caso você esteja à procura de uma casa para alugar sozinho ou com amigos, segue alguns  links de imobiliárias em Sorocaba:

\begin{itemize}
  \item BIS <http://www.bissorocaba.com.br/>
  \item Emaximóvel <http://www.imobiliariaemaximovel.com.br/>
  \item M\&C imóveis <http://www.imobiliariaemsorocaba.com.br/>
  \item Mendes Ortega <http://www.mendesortega.com.br/>
  %\item Ribera <http://www.riberaimoveis.com.br/> %-- site indisponível
  \item Reis imóveis <http://www.reisimoveis.com.br>
  \item Casabranca Imoveis <http://www.casabrancanet.com.br>
\end{itemize}

Abaixo indicaremos lugares onde há grande quantidade de alunos da UFSCar, não necessariamente do curso de Computação, em que, como são apenas endereços, você deve procurar a portaria de cada lugar e perguntar a forma de alugar (sendo ela por imobiliária ou direto com o proprietário), sendo eles:

\begin{itemize}
  \item Vila Universitária
    \begin{itemize}
      \item Travessa direita da Rodovia João Leme dos Santos (depois da UFSCar)
      \item Condomínio de chalés. Lugar bem próximo da UFSCar (especificamente 10 minutos a pé), entretanto, fica do outro lado da rodovia, pode ser um pouco perigoso atravessar em horários de pico. Não há muitos lugares para poder comprar naquela região, por isso, faça a compra do mês. Um bom lugar para ter várias horas de sono e maximizar a economia do dinheiro. O preço varia de R\$600 a R\$800 (depende do proprietário/imobiliária e localização), com o condomínio incluso na grande maioria dos casos, mas você poderia dividir o espaço com 3 pessoas.
    \end{itemize}

  \item Vila Universitária II
    \begin{itemize}
      \item Travessa direita da Rodovia João Leme dos Santos (depois da UFSCar)
      \item O Condomínio Universitário Vila II é composto por apartamentos de 35 m2 (metros quadrados). Localiza-se nas proximidades da UFSCar (aproximadamente 15 minutos, a pé). Recomendado para aqueles que gostam de acordar um pouco mais tarde (isso não significa que você não vai perder a hora). As despesas com o aluguel giram em torno de R\$650, onde já é incluso água, luz, gás e internet (sendo esse último item compartilhado com todos que moram, logo é um pouco lento). É possível dividir o local com mais uma pessoa para aqueles que querem economizar uma grana. No mais, tem uma "salinha de estudos" (onde jogamos baralho) e uma churrasqueira.
    \end{itemize}
  \item Jambalaia
    \begin{itemize}
      \item Estrada Dr. Celso Charuri, 307
      \item Condomínio de kitnets. Lugar também bem próximo da UFSCar, não tão perto quanto a Vila Universitária, tem os mesmos contra e os pŕos da Vila. Existem três tipos de apartamento, com três preços variados. O primeiro tipo, é a kitnet. O preço varia é algo entre r\$500,00 e R\$600,00, mas é impossível dividir com mais alguém. O segundo tipo, é um apartamento de um quarto. Dá pra dividir com uma pessoa e o preço fica entre R\$600,00 e R\$700,00. O último é o apartamento com dois quartos, cujo preço é algo entre R\$700,00 e R\$800. No condomínio tem uma academia, então não ficar MOnSTRO não é desculpa.
    \end{itemize}
  \item Saragoza (condomínio fechado)
    \begin{itemize}
      \item Av. Dr. Armando Pannunzio, 1791-1965
      \item{O lugar fica no final da Av. Armando Pannunzio, próximo de um ponto de ônibus que demora cerca de 20 minutos até a UFSCar, portanto fique atento aos horários do ônibus através do aplicativo da URBES. O bairro contém um McDonalds, mercados e uma galeria onde é possível encontrar pizzaria, açaí, lojas de eletrônicos, padaria e até um banco da CAIXA. 
	      O aluguel é em torno de R\$800,00 para o apartamento com dois quartos, R\$1100,00 o apartamento com três quartos e R\$1300,00 a cobertura (três quartos e dois banheiros); R\$290,00 de condomínio, que inclui a conta de água. Há também um espaço de lazer que tem uma quadra e uma área para churrasco}
    \end{itemize}

  \item Portal dos Bandeirantes (condomínio fechado)
    \begin{itemize}
      \item R. Benedito Venceslau Mendes, 171
      \item O lugar fica no final da Av. Armando Pannunzio, tendo as mesmas características do Saragoza (perto do ponto de ônibus, McDonalds, mercado, etc) O aluguel é por volta de R\$800, sendo R\$200 de condomínio. O espaço de lazer tem como área para churrasco, piscina, quadra, área verde.
    \end{itemize}
  \item Mangueiras (condomínio fechado)
    \begin{itemize}
      \item Rua Orlando Bismara, 130
      \item A descrição do arredores é parecida com do Portal e do Saragoza. No espaço há uma quadra, área para churrasco, área verde com parque para crianças e um ponto de ônibus no mesmo quarteirão do condomínio. O aluguel fica em torno de R\$1200 para todos os apartamentos, sendo que eles têm três quartos --um pequeno, um grande e uma suíte-- e dois banheiros. O condomínio custa R\$270,00.
    \end{itemize}
\end{itemize}

Como última opção, temos a moradia da UFSCar, entretanto,  deixamos na seção Bolsas da UFSCar que se encontra na página \pageref{moradia}.

\subsection{Transporte}
\subsubsection{Indo ou Voltando da Universidade}
Existe uma linha de ônibus em Sorocaba que tem ponto final dentro do nosso campus. O nome da linha é “UFSCAR” e o número é 80. Seu ponto de partida é o terminal São Paulo. 

Os ônibus da cidade de Sorocaba não possuem cobradores. Para utilizar o transporte é preciso possuir passe, que atualmente custa R\$ 3,80. Estudantes podem adquirir passe de ônibus por R\$ 1,50 (Os ônibus da cidade tem um sistema de cartões de passes que precisam ser feitos junto a faculdade - secretária).

Brotip: A Urbes tem um aplicativo com os horários de todos os ônibus que rodam por Sorocaba. É uma boa ter pra não ficar perdendo tempo no ponto quando o próximo ônibus passa só uma hora mais tarde. 

Para saber mais informações sobre a linha de ônibus da UFSCar e sobre como adquirir um cartão de estudante acesse o link \newline{<http://www.urbes.com.br/transporte-horario-onibus>}

Para utilizar o passe de estudante, você deve registrar no site da URBES através do link <https://www.urbes.com.br/Estudantes/>

Outro detalhe importante que deve-se destacar é a possibilidade de integração entre os ônibus de Sorocaba. Isso significa que você não precisaria pagar duas viagens entre dois diferentes ônibus. (UFSCar e 9 de Julho, por exemplo). O tempo máximo da integração fica em: o tempo da viagem do ônibus + 1 hora contando a partir do ínicio da viagem ou então 3 integrações. 

No caso da linha UFSCar, o tempo máximo para integração é de 1 h e 40 min e você pode observar as linhas que estão integradas neste link: \newline{<https://www.urbes.com.br/transporte-onibus-integracao>}

Uma linha alternativa de ônibus que funciona todos os dias e que passa em frente ao campus (fique atento ao ponto de descida caso decisa usá-lo) é o Salto de Pirapora - Sorocaba \newline{<http://saltoemuitomais.blogspot.com.br/p/horario-de-onibus.html>}. Este ônibus aceita dinheiro e o valor atual da passagem é R\$ 4,55.

Brotip: Para aqueles que moram nas repúblicas próximas a faculdade (Vila Universitária, Residencial Flora, Jambalaia, etc), é possível fazer a carteirinha do Piracema e, como consequência, conseguir pagar meia passagem ou até ter o passe livre. Para o passe livre é necessária a comprovação de renda. Mais informações sobre os passes de ônibus podem ser dadas na DiGRa.

LEMBRE-SE QUE: A linha da UFSCar não roda aos domingos e feriados! Então se você decidir morar perto da universidade e quiser passear, vai ter que usar o Piracema. Os passes do Piracema também não são aceitos aos domingos e feriados, ou seja, você tem que pagar o valor integral.

Brotip 2: Agora tem Uber em Sorocaba. Pra ir ou voltar do rolê é uma boa pedida. 

\subsubsection{Voltando para Casa}
A gente sabe que os ônibus tão cada vez mais caros e, pra dar um jeito, foram
criado grupos de carona no Facebook:
\newline{<https://www.facebook.com/groups/298050893559371/?fref=tsnewline>}. Nesse grupo, é possível oferecer caronas, diminuindo seus gastos e ainda ganhando uma companhia, ou procurar caronas, e aí cê vai trocando uma ideia e ainda paga mais barato.

Para algumas cidades maiores, como São Paulo e Campinas, existem grupos específicos de carona no Facebook, basta dar uma procurada no que for mais conveniente pra você.

Se, de jeito nenhum, você achar uma carona e tiver que ir para a rodoviária, fica em paz: existem ônibus que vão direto para a rodoviária saindo de ambos os terminais (São Paulo e Santo Antônio). Além disso, se você morar nas proximidades da faculdade, não precisa ir até o terminal: o UFSCAR passa em frente também. (:  

%%% COMIDA %%%
\begin{multicols}{2}
  [
  \subsection{Comida}
  Embora a comida do RU seja maravilhosa e a gente coma ela todo dia, duas vezes ao dia, cinco dias por semana, às vezes enjoa, né. Por isso segue uma lista de alguns lugares que você poderia ir comer, ou então, pedir para entregar:
  ]

  \begin{itemize}
    \item \textbf{Neri Lanches}
      \newline Av. Dr. Armando Pannunzio, 1077, Vila Espírito Santo, Sorocaba - SP
      \newline \texttt{www.nerilanches.com.br}
      \newline (15) 3222-8136
  \end{itemize}
  \begin{itemize}
    \item \textbf{Habibs}
      \newline Parça Oxford (Av. Dr. Armando Pannunzio), 26, Jardim Europa, Sorocaba - SP
      \newline (15) 3003-2828
  \end{itemize}
  \begin{itemize}
    \item \textbf{Disk Salgados (entregas em Salto de Pirapora e perto da UFSCar)}
      \newline Atendimento das 15h às 22h
      \newline(15) 9 9710-3090 ou (15) 9 9134-1169
  \end{itemize}
  \begin{itemize}
    \item \textbf{Cantinho da Gê} (marmitex e restaurante)
      \newline Av. Gal. Carneiro, 706 - Vila Lucy, Sorocaba - SP
      \newline (15) 3202-4389
  \end{itemize}
  \begin{itemize}
    \item \textbf{Esfiharia e Pizzaria Canalle (entregas somente perto da UFSCar)}
      \newline Rua Ovideo Leme dos Santos, 326 - Centro - Salto de Pirapora
      \newline \texttt{www.pizzariacanalle.com.br}
      \newline (15) 3492-3700/(15) 3292-3990
  \end{itemize}
  \begin{itemize}
    \item \textbf{Seu Batata}
      \newline Rua Aparecida, 754, Jardim Santa Rosália
      \newline \texttt{www.seubatata.com.br}
      \newline (15) 3031-3233
  \end{itemize}
  \begin{itemize}
    \item \textbf{Pizzaria Bortolotto}
      \newline Rua Padre ângelo Sofia, 170, Jardim Paulistano
      \newline \texttt{www.pizzariabortolotto.com.br}
      \newline (15) 3492-2239/(15) 3292-2108
  \end{itemize}
\end{multicols}

%%% DIVERSÃO %%%
\subsection{Diversão}
Para aqueles momentos nos quais você não vai aguentar mais estudar ou olhar pra tela do seu computador, deixamos aqui alguns lugares pra você bater perna e fazer algo diferente pelo menos uma vez no semestre (ou várias vezes, your call).

\begin{multicols}{2}
  [
  \subsubsection{Parques}
  ]
  \begin{itemize}
    \item \textbf{Parque das Águas}
      \newline Av. Dom Aguirre, S/N - Jardim Abaete, Sorocaba - SP
  \end{itemize}
  \begin{itemize}
    \item \textbf{Zoológico Municipal Quinzinhos de Barros}
      \newline R. Teodoro Kaisel, 883 - Vila Hortência, Sorocaba - SP
      \newline (15) 3227-5454
      \newline Ter-Dom das 9h às 17h
  \end{itemize}
  \begin{itemize}
    \item \textbf{Parque dos Espanhois}
      \newline R. Dr. Campos Sales, S/N - Vila Assis, Sorocaba - SP
  \end{itemize}
  \begin{itemize}
    \item \textbf{Parque Natural Municipal da Biquinha}
      \newline Av. Comendador Pereira Inácio, 1112 - Jardim Vergueiro, Sorocaba - SP
      \newline (15) 3224-1997
      \newline Todos os dias das 8h às 17h
  \end{itemize}
\end{multicols}


\begin{multicols}{2}
  [
  \subsubsection{Shoppings}
  ]
  \begin{itemize}
    \item \textbf{Shopping Iguatemi Esplanada}
      \newline Av. Prof. Izoraida Marques Peres, 401 - Campolim, Sorocaba - SP
      \newline Horário: 10h às 22h
      \newline (15) 3219-9900
  \end{itemize}
  \begin{itemize}
    \item \textbf{Shopping Cidade}
      \newline Av. Itavuvu, 3373 - jardim Santa Cecília, Sorocaba - SP
      \newline Horário: 10h às 22h
      \newline (15) 3333-0200
  \end{itemize}
  \begin{itemize}
    \item \textbf{Pátio Ciane Shopping}
      \newline Av. Dr. Afonso vergueiro, 823 - Centro, Sorocaba - SP
      \newline Horário: 10h às 22h
      \newline (15) 3333-3333
  \end{itemize}
  \begin{itemize}
    \item \textbf{Sorocaba Shopping}
      \newline Av. Dr. Afonso vergueiro, 1700 - Centro, Sorocaba - SP
      \newline Horário: 10h às 22h
      \newline (15) 3232-2757
  \end{itemize}
\end{multicols}

\begin{multicols}{2}
  [
  \subsubsection{Bares}
  ]
  \begin{itemize}
    \item \textbf{Video Game Rock Bar}
      \newline R. Aparecida, 675 - Jardim Santa Rosália, Sorocaba - SP
      \newline Horário: Ter-Qua: 18 às 23h Sex-Sab: 18h às 2h
      \newline (15) 3442-8101
  \end{itemize}
  \begin{itemize}
    \item \textbf{Butiquim da Carne}
      \newline Av. Barão de Tatuí, 98 - Vergueiro, Sorocaba - SP
      \newline (15) 99655-9846
  \end{itemize}
  \begin{itemize}
    \item \textbf{Retroid}
      \newline Rua Rio Grande do Sul, 420, Sorocaba - SP
      \newline Horário: 18h às 6h
      \newline Sábado: 11h às 23h Domingo: 11h às 17h
      \newline (15) 3326-6501
  \end{itemize}
  \begin{itemize}
    \item \textbf{Bar do Alemão}
      \newline Av. Eugênio Salerno, 396 - Centro, Sorocaba - SP
      \newline Horário: Ter-Sex: 11h às 15h e 18h às 23h
      \newline Sábado: 11h às 23h Domingo: 11h às 17h
      \newline (15) 3229-9111
  \end{itemize}
  \begin{itemize}
    \item \textbf{The Crown English Pub}
      \newline R. Profa. Francisca de Queiroz, 105 - Vila Indpedência, Sorocaba - SP
      \newline Horário: Seg-qui: 17h às 0h
      \newline (15) 3202-8323
  \end{itemize}
  \begin{itemize}
    \item \textbf{Hangar 51}
      \newline R. Victorio Pegoretti, 51 - Jardim Faculdade, Sorocaba - SP
      \newline Horário: Qua-Sex: 18h às 2h
      \newline Sábado: 11h às 2h
      \newline (15) 3229-8851
  \end{itemize}
\end{multicols}
\begin{multicols}{2}
  [
  \subsubsection{Lazer e Cultura}
  ]
  \begin{itemize}
    \item \textbf{Biblioteca Municipal de Sorocaba}
      \newline R. Ministro Coqueijo Costa, 180 - Alto da Boa Vista, Sorocaba - SP
      \newline Horário: Seg-Sex: 8h às 16:50
      \newline Sábado: 13h às 16:50
      \newline (15) 3228-1955
  \end{itemize}
  \begin{itemize}
    \item \textbf{Museu de Arte Contemporânea}
      \newline Av. Dr. Afonso Vergueiro, 280 - Centro, Sorocaba - SP
      \newline Horário: Seg-Sex: 9h às 17h
      \newline (15) 3233-1692
  \end{itemize}
  \begin{itemize}
    \item \textbf{FUNDEC Sorocaba}
      \newline R. Brig. Tobias, 73 - Centro, Sorocaba - SP
      \newline Horário: Seg-Sex: 8h às 18h
      \newline Sábado: 8h às 12h
      \newline (15) 3233-2200
  \end{itemize}
  \begin{itemize}
    \item \textbf{SESC Sorocaba}
      \newline R. Barão de Piratining, 555 - Jardim Faculdade, Sorocaba - SP
      \newline Horário: Ter-Sex: 9h às 21:30
      \newline Sab-Dom: 10h às 18:30
      \newline (15) 3332-9933
  \end{itemize}
  \begin{itemize}
    \item \textbf{Jardim Botânico de Sorocaba}
      \newline R. Miguel Montoro Lozano - Jardim Iguatemi, Sorocaba - SP
      \newline Horário: Ter-Dom: 9h às 17h
      \newline (15) 3227-9996
  \end{itemize}
\end{multicols}

Tem mais informações no link a seguir. Desde bares até os shows que acontecem. Sempre tem coisa legal rolando por Sorocaba, então não custa nada entrar pra dar uma olhadinha. <http://agendasorocaba.com.br/>

\begin{multicols}{2}
  [
  \subsubsection{Postos/Hospitais}
  Nesta seção colocaremos um compilado de postos de saúde e hospitais, que porventura precise, esse tipo de informação é sempre bom saber de antemão.
  Lembrando que todos os endereços que colocamos aqui são os mais próximos da Av General Carneiro por motivos de facilidade.
  ]
  \begin{itemize}
    \item \textbf{Hospital Evangélico de Sorocaba}
      \newline Av. General Carneiro, 465
      \newline (15) 2101-6600
  \end{itemize}
  \begin{itemize}
    \item \textbf{UPH Zona Leste Sorocaba}
      \newline R. Cel. Nogueira Padilha, 2585 - Vila Hortência
      \newline (15) 3333-0100
  \end{itemize}
  \begin{itemize}
    \item \textbf{Pronto Atendimento UPH}
      \newline Av. General Carneiro, 1670
      \newline (15) 3202-2495
  \end{itemize}
  \begin{itemize}
    \item \textbf{UNIMED}
	\newline R. Antonia Dias Petri, 135 - Pq. Sta Isabel
	\newline (15) 3229-3000
	\newline www.unimedsorocaba.coop.br
  \end{itemize}
\end{multicols}

\begin{multicols}{2}
  [
  \subsubsection{Gás/Água}
    Só realmente sabemos a falta que o gás faz quando ele acaba, então pra dar uma ajuda pra vocês, deixamos aqui alguns números pra vocês não passarem por esse perrengue:
  ]
  \begin{itemize}
    \item \textbf{Disk Gás Sorogás}
      \newline R. Doutoer Américo Figuereido, 576, Jardim Simus
      \newline (15) 3221-1549
  \end{itemize}
  \begin{itemize}
    \item \textbf{Depósito de Gás e Água Jardim São Paulo}
      \newline R. Dr. Benedito Cardoso Franco, 61, Vila Espírito Santo
      \newline (15) 3221-6897
  \end{itemize}
  \begin{itemize}
    \item \textbf{Disk Água General}
      \newline Av. General Carneiro, 218
      \newline (15) 3242-7800/(15) 3011-7201
  \end{itemize}
  \begin{itemize}
    \item \textbf{Almeida Gás e Água}
      \newline Rua Encarnação, 289, Wanel Ville
      \newline (15)3221-5636/(15)3011-3103
  \end{itemize}
  \begin{itemize}
    \item \textbf{CIA Gás}
      \newline Rua Leo Migliori, 51, Jardim Wanel Ville IV
      \newline (15) 3013-3949/(15) 3217-8110
  \end{itemize}
  \begin{itemize}
    \item \textbf{Disk Água General}
      \newline Av. General Carneiro, 218
      \newline (15) 3242-7800/(15) 3011-7201
  \end{itemize}
\end{multicols}


%%% ESTRUTURA/SERVICOS DA UNVIERSIDADE %%%
\subsection{Estrutura/Serviços da Universidade}

\subsubsection{Biblioteca}
\noindent Funcionamento de segunda a sexta
\begin{itemize}
  \item Expediente das 8h às 22h
  \item Emprestimo e devolução de livros das 8h às 21h45
\end{itemize}
\noindent Qualquer pessoa pode entrar e ler os livros dentro do prédio, porém para os empréstimos é necessário um cadastro na biblioteca, que é feito na própria biblioteca em um período determinado. Mais informações, consulte o site da B-So:
\texttt{<http://www.sorocaba.ufscar.br/bso/>}

\subsubsection{Laboratórios}
Temos quatro laboratórios, sendo três para uso específico (Lab. SO.,Lab. Circuitos, Lab. Redes) e um para uso geral (LEC), sendo que todos eles estão concentrados no ATLab (Prédio Roxo). Para utiliza-lós, basta que você procure o Técnico Administrativo Thiago para a utilização no período diurno (8h-18h).

Fora esses, ainda há mais dois laboratórios que ficam no último andar do prédio AT2 (Prédio Vermelho).

\subsubsection{Restaurante Universitário (RU)}
O RU é o Restaurante Universitário que temos no nosso campus. De segunda à sexta é servido aos alunos almoço e janta. Para utilizar o RU basta comprar o ticket do RU (a venda normalmente acontece do lado de fora do ambulatório) e entregá-lo na entrada.

Funcionamento de segunda à sexta: 11h às 13h30 e 17h30 às 19h

No sábado: 11h às 13h30

Tickets vendidos de segunda à sexta: 11h às 13h e 17h30 às 19h00

Valor: R\$1,80 à alunos não bolsistas.

Quer saber sobre o cardápio do RU? Acesse o link

\texttt{<http://www.sorocaba.ufscar.br/ufscar/index.php?pg\_id=39>}

\subsubsection{Ambulatório}
Na UFSCar Sorocaba temos o atendimento de enfermagem ambulatorial (verificação da pressão arterial, da temperatura e etc), médico e também psicológico. Para receber o atendimento de enfermagem ambulatorial basta comparecer pessoalmente e apresentar a carterinha estudantil.

Já para os outros atendimentos, você deve marcar uma consulta pessoalmente ou pelo telefone, seguem eles abaixo:

\noindent \textbf{Médico:}
\newline Dr. LUIZ FERRAZ DE SAMPAIO NETO (ginecologista)
\newline (15) 3229-5918
\newline lfsampaio@ufscar.br

\begin{multicols}{2}
\noindent \textbf{Psicóloga:}
  \newline FABIANA MIDORI OIKAWA
  \newline (15) 3229-5925
  \newline fbkawa@ufscar.br

\noindent \textbf{Enfermeira:}
  \newline SANDRA REGINA ROCHA ARAUJO
  \newline (15) 3229-5918
  \newline sandra@ufscar.br

\end{multicols}

\subsubsection{Xerox}
O seviço de xerox e cópias fica localizado próximo ao prédio do RU, na área de vivência, o funcionamento acontece de segunda a sexta e o expediente é das 8h às 21h. 

O e-mail para envio de arquivos para impressão é: \texttt{ufscar.xerox@gmail.com}

\subsubsection{Bolsas}
A UFSCar possui um sistema de bolsas para ajudar alunos em vulnerabilidade socioeconômica, esta é uma das formas oferecidas aos estudantes para que estes possam permanecer na universidade, afinal entrar na universidade é apenas o primeiro passo de uma longa jornada!

A página da Proace (Pró reitoria de Assuntos Comunitários e Estudantis) \texttt{<http://www.proace.ufscar.br/bolsa-e-auxilio-para-estudantes>} possuem mais informações sobre as bolsas e auxílio aos estudantes.

Você aluno da computação que ficou interessado também pode procurar pela assistente social do nosso campus.

\paragraph{Moradia}\label{moradia}
A UFSCar campus Sorocaba não possui moradia dentro da universidade, neste caso a universidade aluga casas para oferecer bolsa moradia aos alunos que precisam.

Para os bolsistas dessa modalidade, além do aluguel, a universidade também é responsável por diversos gastos como água, luz, IPTU e gás. 

Para saber mais sobre esta bolsa e quem tem direito à ela acesse o link \texttt{www.proace.ufscar.br/bolsa-e-auxilio-para-estudantes-1/bolsa-moradia}

\paragraph{Alimentação}
Os alunos que possuem bolsa alimentação têm direito à duas refeições diárias (almoço e janta) no RU gratuitamente, além de uma marmita nos sábados. 

Maiores informações sobre esta bolsa podem ser encontradas no link <http://www.proace.ufscar.br>

\paragraph{Atividade}
Este é um tipo de bolsa onde o aluno desenvolve uma atividade 8 horas por semana durante oito meses (4 meses no primeiro semestre e 4 meses no segundo) e recebe uma ajuda de custo pela atividade desempenhada.

O link para se obter mais informações é:
\newline <http://www.proace.ufscar.br/bolsa-e-auxilio-para-estudantes-1/bolsa-atividade>

\subsection{Como Estudar}
Ao entrar na faculdade, geralmente temos a impressão de que vai ser igual ao ensino médio ao menos bastante parecido. Eis o que nós veteranos podemos afirmar com convicção: não é nem um pouco parecido. Então pra dar aquela ajudada, separamos algumas dicas de estudo que aprendemos na marra.

\subsubsection{Não deixe pra última hora}
Essa dica parece meio boba, mas a real é que a gente no começo sempre acha que vai dar tempo. Evite fazer isso. Diferente do ensino médio, o conteúdo não costuma vir mastigado e, às vezes, é necessário bastante estudo extra-classe. Portanto estude com antecedência sempre que possível (às vezes realmente não é)! Isso te ajuda em vários aspectos: você tem tempo para tirar dúvidas com o seu professor ou monitor, você pode revisar a matéria um dia antes da prova só pra conferir tudo e você tem mais chances de ter uma boa noite de sono antes da prova.

\subsubsection{Vá nas monitorias e no atendimento do professor}
O nosso curso tem bastante disciplina prática, o que significa que passamos bastante tempo programando e mais tempo ainda olhando pra tela do computador pensando em como resolver um problema que aparece. Em alguns casos, você não vai conseguir achar o erro de jeito nenhum e é pra isso que servem os atendimentos e as monitorias: ninguém melhor do que alguém que já fez a disciplina para te ajudar. 

Aliás, não é somente nas disciplinas práticas que você às vezes esbarra nas dificuldades: as disciplinas teóricas também apresentam algumas surpresas. Pra nossa sorte, elas também têm atendimento dos professores e/ou monitoria! 

Não tenha vergonha e vá perguntar! A gente tá aqui na faculdade pra aprender mesmo e eles tão na faculdade pra ajudar a gente e nos indicar qual o melhor caminho para resolver um problema.

\subsubsection{Faça as listas}
Essa dica parece besta e em alguns momentos simplesmente não dá tempo de fazer as listas, mas, sempre que der, faça. Para as disciplinas teóricas é um excelente modo de você aprender a colocar no papel o que você só saberia explicar falando (inclusive isso é um baita treinamento pra hora da prova). Para as disciplinas práticas, é o momento de você ver funcionando os conceitos aprendidos.

Brotip: alguns professores baseiam provas nas listas de exercícios, portanto não fique vacilando achando que nada da lista cai porque tem conteúdo porque cai sim.

Brotip 2: alguns professores \textbf{escolhem} uma questão da lista pra por na prova. Uma questão faz muita diferença.

\subsubsection{Fale com seus veteranos}
Procure gente que já fez a disciplina e pergunte qual a melhor forma de estudar. Cada professor tem um estilo de prova e é bem legal quando a gente não é surpreendido por um tipo de questão que não estávamos esperando. Fora isso, muitos veteranos têm uma memória ótima e podem lembrar mais ou menos do conteúdo da prova que ele fez quando cursou a disciplina. 

\subsubsection{Não se desespere}
Cada um tem o próprio jeito que funcionar que dá certo. Essas dicas são bem genéricas e não são lei. Tem gente que acha muito melhor estudar cinco minutos antes da prova e tem gente que com duas semanas de antecedência já está pegando o livro na biblioteca. Se nenhuma delas der certo pra você, significa que você não achou o seu jeito ainda. 

Só não desanime! 

No começo é difícil pra todo mundo, mas logo você pega o jeito. 

\section{VIDA FORA DA ACADEMIA}
Sim, existe vida fora da academia e não pense em ser frangão nessa parte, bixo. A universidade e os próprios alunos oferecem outros tipos de atividades pra você interagir com as pessoas do seu curso e fora também. A participação não é obrigatória, mas é sempre muito legal comparecer (e às vezes te rende uns créditos extras).

\subsection{Centro Acadêmico}
O C.A. é uma entidade estudantil constituida pelos alunos do mesmo curso, tem como objetivo levar a demanda do curso perante os conselhos que existem dentro da universidade, isso tudo, para lutar pela melhoria do curso; seja opinando nas ementas do curso, no projeto pedagógico e também na infraestrutura da universidade. Essa entidade também é responsável por criar eventos com caráter social, este manual é um grande exemplo disso. No nosso caso, que é o curso de Ciência da Computação, nosso Centro Acadêmico é o < C.A.C.C.S / Pata do Bisão >, nesse momento estamos com 10 membros ativos da gestão de 2016/2. Temos diversos projetos e estamos sempre querendo membros para poder ajudar neles! Se você tiver uma ideia muito bacana e/ou queira ajudar, converse conosco! Participem de nossas reuniões para que possamos cada vez melhorar nosso curso! Você pode nos contactar pela página do Facebook: \newline <http://www.facebook.com/CACCS.UFSCar/>

\subsection{Atlética Raça Brisão}
A atlética é uma entidade organizada por estudantes dentro das universidades. Nosso campus possui uma Atlética Geral e nosso curso também possui uma atlética, a “Atlética Raça Brisão”, responsável pela realização de eventos esportivos, culturais e sociais, dentre outras atividades.

\subsection{Semana da Computação}
A SeCoT (Semana de Computação e Tecnologia) é um evento organizado pelos alunos do curso que tem como objetivo difundir em palestras e/ou workshops, novas tecnologias, metodologias e discussões que permeiam a computação e inovação. Você pode ver algumas fotos das edições passadas na página do Facebook: \newline <https://www.facebook.com/secot.ufscar/>

\subsection{Mini Maratona de Programação}
A Mini Maratona de Programação é o evento que encerra a SeCoT. É feita nos moldes da famosa Maratona de Programação da ACM-SBC, onde grupos de três pessoas enfrentam uma sequência de problemas desafiantes de programação.

\subsection{Esportes}
A Atlética Geral oferece treinos de diversas modalidades esportivas, tais como Basquete, Futebol, Vôlei e Handball. Também ocorrem amistosos contra as outras faculdades de Sorocaba.

Dentro da UFSCar ainda há o Interbixos, que é o campeonato entre os cursos com times apenas de calouros, e o Intercursos. Ambas são ótimas oportunidades pra interagir com seus colegas de classe e de curso. A página do Facebook é: <https://www.facebook.com/atleticaufscarsorocaba/>

\subsection{Bateria Chapelaria}
A Bateria Chapelaria procura representar nossa universidade em eventos, tanto dentro quanto fora do ambiente universitário. Seja animando festas, jogos ou mesmo chamando a atenção para assuntos de caráter político e de interesse dos alunos em geral.

A bateria é aberta para todos, são promovidas escolinhas e oficinas para os quem tem vontade de aprender a tocar algum instrumento. Segue a página do Facebook: <https://www.facebook.com/bateriachapelaria>

\subsection{Empresa Junior Beets}
A Beets é uma empresa júnior gerida e integrada por estudantes do curso de Ciência da Computação da UFSCar - Sorocaba, a qual, contribui com o desenvolvimento de soluções de TI da região, além de propiciar aos seus membros a oportunidade de vivência e aperfeiçoamento profissional. Para mais informações, tem a página no Facebook: <https://www.facebook.com/beetsjr/>

\subsection{Iniciação Científica (IC)}
A iniciação científica (IC) é uma atividade de pesquisa desenvolvida por alunos da graduação com o apoio de um professor orientador que seja da área em que o trabalho é realizado.

Esta atividade é uma ótima forma dos alunos terem um primeiro contato com a pesquisa científica, uma vez que a maioria dos alunos não possuem experiência e terão o apoio e acompanhamento de um orientador.

A iniciação científica pode ser feita com ou sem bolsa, tudo dependerá do seu projeto ser aprovado por uma agência financiadora,  como por exemplo CNPq (Conselho Nacional de Desenvolvimento Científico e Tecnológico) e FAPESP (Fundação de Amparo à Pesquisa do Estado de São Paulo). 

Se você tem interesse fazer uma iniciação científica, fique atento às chamadas para interessados em IC. Nossos professores divulgam com frequência oportunidade de trabalhos para os alunos que desejam fazer pesquisa. Ou então você pode apresentar à algum professor uma ideia de trabalho que você tenha para que ele possa te orientar.

\subsection{Monitoria}
A Monitoria é uma atividade extracurricular em que o aluno se encarrega em auxiliar o professor de uma determinada matéria, sanando dúvidas dos alunos que comparecerem ao horário de atendimento e passando trabalhos extraclasses; sendo possível essa atividade ser realizada com bolsa ou voluntariamente.

Todas as monitorias valem quatro créditos, sendo 12 horas reservadas semanalmente para esta atividade.

\subsection{Share Centro de Línguas}
A entidade Share é um Centro de Línguas da UFSCar que tem como objetivo incentivar o compartilhamento de línguas, no qual os alunos aprendem idiomas através das experiências linguísticas adquiridas pelos professores. As aulas são gratuitas.
Mais informações vocês podem encontrar na página do Facebook:

\texttt{https://www.facebook.com/ShareLinguas/}
